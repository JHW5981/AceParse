%%%%%%%%%%%%%%%%%%%%% Page 1 %%%%%%%%%%%%%%%%%%%%%
[Title + Name + Affiliation]


%%%%%%%%%%%%%%%%%%%%% Page 2 %%%%%%%%%%%%%%%%%%%%%
[Introduction]
“it is also important to incorporate the contextual information into the recommendation process in order to recommend items to users under certain circumstances”
				---- From RS Handbook (2022)

For example,
Such circumstance can be temporal information, or location that independent with users or items, but can be reflected from their interaction records in some way.
(Context, User) => item
(Winter, The User Interacted with Sporting Equipment) => Skiting Shoses
Common sense: Winter usually has snow

Prerequisite is also a certain kind of knowledge circumstance that constrains user’s action.
- Context: The stage of an primitive/advanced level of knowledge
- User: mostly interacted with statistic-based course
- Common sense: the prerequisite relation of knowledge, .e.g, Naïve Bayes is the pre-conditional constrain of Bayes Classifier.
% For example, using the temporal context, a travel recommender system would provide a vacation recommendation in the winter that can be very different from the one in the summer. Similarly, in case of personalized content delivery on a website, a user might prefer to read world news when she logs on the website in the morning and the stock market report in the evening, and on weekends to read movie reviews and do shopping, and appropriate recommendations should be provided to her in these different contexts.

%%%%%%%%%%%%%%%%%%%%% Page 3 %%%%%%%%%%%%%%%%%%%%%
[Introduction]
Prerequisite is a term that usually used in Educational-related Domain.

Prerequisite is a certain kind of knowledge circumstance that constrains user’s action.
- If a person do not have the knowledge about Bayes, then we should not recommend Bayes Classifier to him/her.


%%%%%%%%%%%%%%%%%%%%% Page 4 %%%%%%%%%%%%%%%%%%%%%
[Introduction]
> Why prerequisite is helpful (for recommender)?
    : A case

%%%%%%%%%%%%%%%%%%%%% Page 5 %%%%%%%%%%%%%%%%%%%%%
[Problem Definition]
> To align with the general CARS, our PDRS is formulated as ..


%%%%%%%%%%%%%%%%%%%%% Page 6 %%%%%%%%%%%%%%%%%%%%%
[Data Construction]
> Keyword extraction.


%%%%%%%%%%%%%%%%%%%%% Page 7 %%%%%%%%%%%%%%%%%%%%%
[Data Construction]
> 2 branches to build such inference unbalance.


%%%%%%%%%%%%%%%%%%%%% Page 8 %%%%%%%%%%%%%%%%%%%%%
[Approach]
> Two-tower MLP


%%%%%%%%%%%%%%%%%%%%% Page 9 %%%%%%%%%%%%%%%%%%%%%
[Experiment]
Main Results + Ablation


%%%%%%%%%%%%%%%%%%%%% Page 10 %%%%%%%%%%%%%%%%%%%%%
[Discussion 1,2,3,4]

