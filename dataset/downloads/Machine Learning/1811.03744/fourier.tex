% !TEX root =  main.tex

% \section{Fourier transform of high-dimensional distributions} \label{ap:fourier}


While it is possible 
to do Fourier analysis over $\mathbb{R}^d$, 
in this paper, we will only do Fourier analysis for functions $f \in  L_1([-1,1]^d)$.  
% We first recall the definition of the Fourier transform of a function over $\mathbb{R}^d$:
\begin{definition}
For any function $f \in L_1([-1,1]^d)$,
% and $\xi \in \mathbb{R}^d$,
we define
$\widehat{f}: \mathbb{R}^d \rightarrow \mathbb{C}$
by
$
\widehat{f}(\xi) = \int_{x \in \mathbb{R}^d} f(x) \cdot e^{ \pi i \cdot \langle \xi, x \rangle} dx.$
\end{definition}

Next, we recall the following standard claims about Fourier transforms of functions, which may be found,
for example, in \citep{smith1995handbook}.
\ignore{\pnote{I did not find the following in {\em Fourier Analysis}, by K\"orner.  
Some of them are in \citep{smith1995handbook}, but, honestly, I'm only guessing that the others are there.  Do either of you have
a reference handy that we could use? \red{Rocco:  Are we happy with this now?}}
}
\begin{claim}~\label{clm:convolution}
For $f,g \in L_1([-1,1]^d)$  let   
$
{h}(x) = \int_{y \in \mathbb{R}^d} f(y) \cdot g(x-y) dy
$ denote the convolution $h = f \ast g$ of $f$ and $g$.
Then for any $\xi \in \mathbb{R}^n$, we have $\widehat{h}(\xi) = \widehat{f}(\xi) \cdot \widehat{g}(\xi)$. 
\end{claim}

Next, we recall Parseval's identity on the cube.

\begin{claim}[Parseval's identity]~\label{clm:Parseval}
For $f: [-1,1]^d \rightarrow \mathbb{R}$ such that $f \in L_2( [-1,1]^d)$, it holds that
$
 \int_{ [-1,1]^d} f(x)^2 dx = \frac{1}{2^d} \cdot \sum_{\xi  \in \mathbb{Z}^d} |\widehat{f}(\xi)|^2. 
$
\end{claim}
The next claim says that the Fourier inversion formula can be applied to any sequence in $\ell_2(\mathbb{Z}^d)$ to obtain a function whose Fourier series is identical to the given sequence. 
\begin{claim}[Fourier inversion formula] \label{clm:inversion}\
For any $g: \mathbb{Z}^d \rightarrow \mathbb{C}$ such that $\sum_{\xi \in \mathbb{Z}^d} |{g}(\xi)^2| <\infty$, the function
% $h: [-1,1]^d$, 
$
h(x) = \sum_{\xi \in \mathbb{Z}^d} \frac{1}{2^d} \cdot g(\xi) \cdot e^{ \pi i \cdot \langle \xi, x \rangle},
$
is well defined and satisfies $\widehat{h}(\xi) = g(\xi)$ for all $\xi \in \mathbb{Z}^d$. \end{claim}
We will also use Young's inequality: 
\begin{claim}[Young's inequality] \label{clm:Young}
% \begin{claim} \label{clm:Young}
Let $f \in L_p([-1,1]^d)$, $g \in L_q([-1,1]^d)$, $1 \le p,q ,r \le \infty$, 
such that $1 + 1/r = 1/p + 1/q$. Then $\Vert f \ast g \Vert_r \le \Vert f \Vert_p \cdot \Vert g \Vert_q$. 
\end{claim}

\subsection{A useful mollifier} \label{sec:mollifier}


Our algorithm and its analysis require the existence of a compactly supported distribution with fast decaying Fourier transform. Since the precise rate of decay is not very important, 
% let us define the $C^{\infty}$ function $b: [-1,1] \rightarrow \mathbb{R}^+$ as fo% llows: 
we use the $C^{\infty}$ function $b: [-1,1] \rightarrow \mathbb{R}^+$ as follows: 
\begin{eqnarray}~\label{eq:b}
b(x) = 
\begin{cases}
c_0 \cdot e^{-\frac{x^2}{1-x^2}} &\textrm{if } |x|<1 \\
0 &\textrm{if } |x|=1.
\end{cases}
\end{eqnarray}
Here $c_0 \approx 1.067$ is chosen so that 
% $\int_{[-1,1]} b(x) dx=1$
$b$ is a pdf;  by symmetry, its mean is $0$.
(This function has previously been used as a
mollifier \citep{kane2010exact,diakonikolas2010bounded}.)
The following fact can be found in \cite{johnson2015saddle} (while it is proved only for $\xi \in \mathbb{Z}$, it is easy to see that the same proof holds if $\xi \in \mathbb{R}$). 
%are easy to verify:\ignore{ (see e.g. \red{[cite]}
% \begin{proposition}~\label{prop:b-properties}
% $b(\cdot)$ defines a probability distribution with mean $0$.
% %  and variance $\Omega(1)$. 
% \pnote{Provisionally deleted ``and variance $\Omega(1)$'' because I cannot see what the variance is independent of.}
% \end{proposition} }
\ignore{\pnote{How is the following fact proved? \red{Rocco:  I think I missed part of the discussion on this on the recent phone call --- were we going to cite a paper of Jelani's or something?}}}
\begin{fact}~\label{eq:FTB}
For $b:[-1,1] \rightarrow \mathbb{R}^+$ defined in (\ref{eq:b}) and $\xi \in \mathbb{Z} \setminus \{0\}$,  we have that $|\widehat{b}(\xi)| \le e^{-\sqrt{|\xi|}} \cdot |\xi|^{-3/4}$. 
\end{fact}

Let us now define the function $b_{d,\gamma}: \mathbb{R}^d \rightarrow \mathbb{R}^+$ as 
$b_{d,\gamma}(x_1, \ldots, x_d) = \frac{1}{\gamma^d} \cdot \prod_{j=1}^d b(x_j/\gamma).$
Combining this definition and Fact~\ref{eq:FTB}, we have the following claim:\ignore{\rnote{Sorry to be dense but I don't see exactly how we get this from the definition of $b_{d,\gamma}$ and Fact~\ref{eq:FTB}}
\anote{Rocco, I think the idea is that since $b(\cdot)$ integrates to $1$, its Fourier transform at any point is bounded by $1$. $b_{d,\gamma}$ is just the product distribution where each coordinate is $b$ (after a translation by $\gamma$).}}
\begin{claim}\label{clm:Fourier-b}
For $\xi \in \mathbb{Z}^d$ with $\Vert \xi \Vert_\infty \ge t$, we have $|\widehat{b_{d,\gamma}}(\xi)| \le e^{-\sqrt{\gamma \cdot t}} \cdot (\gamma \cdot t)^{-3/4}$. 
\end{claim}
The next fact is immediate from (\ref{eq:b}) and the definition of $b_{d,\gamma}$:
\begin{fact}~\label{fact:b-sup-density}
$\Vert b_{d,\gamma} \Vert_\infty = (c_0/\gamma)^d$ and as a consequence,
 $\Vert b_{d,\gamma} \Vert_2^2 \le  (c_0/\gamma)^{2d}$.
\end{fact}


