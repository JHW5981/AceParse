\section{Note}
Include Yet:
\begin{itemize}
    \item Originally, we highlighted that the model is SOTA -> Currently, I'm telling the story as a wrok to full-fill the research gap. Because I find my proposed model can not beat the KGCN, on of the KG-RS SOTA, constructing a lot KG message process and is more elegant at handling KG. Thus, our main contribution lies in introducing prerequisite, a special kind of knowledge, into RS, rather than letting RS become a SOTA model by using additional information.
    Regarding this, I've updated Abs/Intro/Approach
    \item I changed Preliminary(Original) to a re-definition of this problem, i.e. we define a new type of problem -- i.e., PDR problem aiming to recommend the ideal item to user logically. And our model, that is PDRS, is just an instantiation of the task. Regarding this, I've updated PDR Task.
    \item Regarding the insufficient discussion of prerequisite, I have included a separate study of the prerequisite role in the discussion, adding RQ2 to discuss how prerequisite approachable to downstream RS. Regarding this, I've updated PDR Task.

\end{itemize}

Not Include Yet, And Doubt Whether to Include Them:
\begin{itemize}
    \item Compared to normal KG, What is the superiority of PG (Prerequisite Graph)?
    From Liangming: KG is a general objective knowledge, but PG has some guidance for RS, it is a subjective one.
    
    \item To mention the sequential recommender somewhere as a future work.
    \item Related Work is not close enough to the new story.
\end{itemize}


Story from liangming:
\begin{itemize}
    \item The drawbacks of noraml CF -> Why commonsense knowledge is needed -> Exsisting work -> Gap: they are focusing on normal KG -> the drawbacks of normal KG [?] -> Gap: Prerequisite Graph is more appropriate for recommendation, and it is neglected -> PG v.s. KG (Giving examples) -> Purpose
\end{itemize}