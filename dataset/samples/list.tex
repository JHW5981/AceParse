\documentclass{article}
\usepackage{amsmath}
\usepackage{xcolor}
\usepackage{booktabs}
\usepackage{graphicx}
\usepackage{algorithm}
\usepackage{algorithmic}

\begin{document}
\pagenumbering{gobble}

\begin{itemize}
    \item In \textit{ATSumm-2A}, we use the T5-small model proposed by Raffel et al.~[39] to create an abstractive summary of the input tweets.
    \item In \textit{ATSumm-2B}, we use the BART model proposed by Lewis et al.~(lewis2020bart) to create an abstractive summary of the input tweets.
    \item In \textit{ATSumm-2C}, we use the PEGASUS model proposed by Zhang et al.~[54] to create an abstractive summary of the input tweets.
    \item In \textit{ATSumm-2D}, we use the Longformer model proposed by Beltagy et al.~[65] to create an abstractive summary of the input tweets.
    \item In \textit{ATSumm-2E}, we use the ProphetNet model proposed by Qi et al.~(qi2020prophetnet) to create an abstractive summary of the input tweets.
    \item In \textit{ATSumm-2F}, we use the PGN model proposed by See et al.~(see2017get) to create an abstractive summary of the input tweets.
    \item In \textit{ATSumm-2G}, we use the proposed AuxPGN model without considering the key-phrase scores to create an abstractive summary of the given input tweets.
\end{itemize}

\begin{lemma}
  Under Assumption~(fixed joint), the class-generalization error  in definition~(class gen) is given by
\end{lemma}


\end{document}